\chapter{Materiais e Métodos}\label{cap:ferramentas}

A ferramenta OpenCV \citeonline{itseez2015opencv} é uma biblioteca de código aberto focada em problemas de visão computacional em tempo real, desenvolvida pela intel e posteriormente pela Itseez, com suporte a múltiplas plataformas e uso gratuito sobre a licença de código aberto BSD. A ferramenta apresenta suporte a frameworks de aprendizado profundo, como TensorFlow, Pytorch e Caffe \citeonline{wiki:OpenCV} e contempla tanto funções básicas, para aplicações como processamento de imagem, alteração de cor ou resolução, até aplicações avançadas, como detecção facial, identificação de características e biometria.

Neste trabalho, será utilizada a ferramenta de detecção de faces utilizando um classificador de características Haar cascade...

Object Detection using Haar feature-based cascade classifiers is an effective object detection method proposed by Paul Viola and Michael Jones in their paper, "Rapid Object Detection using a Boosted Cascade of Simple Features" in 2001. It is a machine learning based approach where a cascade function is trained from a lot of positive and negative images. It is then used to detect objects in other images.

https://docs.opencv.org/4.1.0/d7/d8b/tutorial_py_face_detection.html

\section{Considerações Finais}

\lipsum[23]
