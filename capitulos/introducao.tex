% ----------------------------------------------------------
% Introdução 
% Capítulo sem numeração, mas presente no Sumário
% ----------------------------------------------------------

\chapter*[Introdução]{Introdução}
\addcontentsline{toc}{chapter}{Introdução}

Reconhecimento facial é uma tarefa trivial para humanos e há décadas tem sido um desafio para visão computacional e aprendizado de máquina, segundo a referência \citeonline{Zhao:2003:FRL:954339.954342}, desde os anos 90 o tema emerge em diferentes conferencias e com o aumento do poder computacional dos dias atuais, sua capacidade se expande muito, fazendo com que tal assunto receba enorme atenção, principalmente devido ao seu grande valor comercial e as mais diversas aplicações possíveis, como verificação de identidade, controle de acesso, segurança, investigação de imagens em bancos de dados, vigilância, entretenimento ou realidade virtual. \cite{appli2014} \cite{Zhao:2003:FRL:954339.954342}

O processo de reconhecimento facial de forma automatizada é separado em 4 principais etapas, conforme detalhado no livro \citeonline{Li:2011:HFR:2073486}, primeiramente deve ser feita a \textit{detecção facial}, que consiste em validar e localizar a existência de alguma face na imagem ou video, a segunda etapa consiste no \textit{alinhamento facial}, para que todas faces da base de dados sigam o mesmo padrão, a terceira etapa é a \textit{extração de características} que permite a obtenção de informação efetiva que será útil na distinção das diferentes faces, a quarta e última etapa consiste na \textit{correspondência de características}, onde as características extraídas anteriormente são comparadas com outras já conhecidas para que sejam identificadas.

Aprofundando o estudo da primeira etapa, de \textit{detecção facial}, a referência \citeonline{faceDetection2001} indica duas diferentes metodologias, a primeira baseada em características e a segunda baseada em imagens, ambas posteriormente podem ser separadas em diversas técnicas mais específicas, como por exemplo a análise de características por constelação ou a análise de imagens com redes neurais, onde cada técnica espécifica possui seus prós e contras em relação as demais.

\section{Objetivos e Motivação}

Este trabalho tem como objetivo encontrar a melhor forma de atuar sobre a primeira etapa (\textit{detecção facial}) e a segunda etapa (\textit{alinhamento facial}) do processo de reconhecimento facial, avaliando o desempenho qualitativo e quantitativo de diferentes metodologias e ferramentas disponíveis e permitindo a rápida identificação de imagens que não possuem uma face, para satisfazer a necessidade descrita a seguir.

Atualmente empresas e órgãos públicos possuem a necessidade de manter cadastros pessoais mas existe grande demanda para que estes cadastros sejam feitos de forma totalmente virtual pela população, pois isso evita o deslocamento de pessoas até os pontos de cadastro e torna todo o processo muito mais ágil. Certos cadastros incluem fotos de identificação e isto traz a necessidade de uma verificação feita por humanos para validar se a mesma consiste em uma foto de face frontal, conforme é necessário para o cadastro.

A validação citada já ocorre nas empresas e órgãos públicos que precisam coletar documentos de forma virtual e é feita de forma totalmente manual, onde funcionários tem que verificar cada uma das imagens recebidas e muitas vezes se deparam com fotos sem nenhuma face frontal ou sem condições de serem identificadas (desfocadas, por exemplo), que são rejeitadas para que uma nova imagem seja solicitada. Estas imagens claramente inválidas por estarem em desacordo com o padrão esperado (foto de face frontal), poderiam facilmente ser eliminadas por uma filtragem anterior, reduzindo grande parte do trabalho que é feito hoje manualmente.