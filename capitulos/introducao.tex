% ----------------------------------------------------------
% Introdução 
% Capítulo sem numeração, mas presente no Sumário
% ----------------------------------------------------------

\chapter[Introdução]{Introdução}

Em empresas que trabalham com logística, a previsão de demanda consiste na análise de dados, principalmente volumes de vendas anteriores, com o objetivo de prever volumes de vendas futuros. A confiabilidade dessas decisões é importante porque elas podem ser usadas para tomar decisões quanto a 

Sendo assim uma previsão imprecisa pode levar a prejuízos por perda de oportunidade ou superprodução caso sejam estocados produtos muito acima ou abaixo da demanda real. Essa tarefa geralmente é feita por um gestor de vendas que coletam e analisam dados históricos, tomando então decisões de acordo com os resultados de suas análises e seus julgamentos e intuições.

Graças ao desenvolvimento das tecnologias de informação, atualmente é possível ter acesso praticamente instantâneo a uma quantidade enorme de dados, de forma que se torna inviável processar esses dados manualmente. Sendo assim, surgiu a oportunidade de processar esses dados de forma automática para gerar previsões cada vez mais confiáveis que levam a decisões mais assertivas.

\section{Objetivos e Motivação}

O objetivo deste trabalho será fazer uma comparação de performance de diferentes modelos de predição em datasets com dados temporais de vendas. Esses modelos terão variedade em sua complexidade, abrangendo cálculos estatísticos simples, filtros adaptativos, e modelos de aprendizado de máquina e serão testados em dados artificiais e reais para comparação. Os dados utilizados.

Serão utilizados como dados séries temporais que indicam o volume de vendas de um produto combinados com outras séries temporais que representam dados externos à empresa como indicadores econômicos, meteorológicos, eventos sazonais e dados do governo. Será feita uma comparação entre esses modelos, considerando o efeito da complexidade do sistema na qualidade das predições.