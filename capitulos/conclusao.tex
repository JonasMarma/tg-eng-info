\chapter{Conclusões e Trabalhos Futuros}\label{cap:conclusao}

\section{Conclusões}

Relembrando o objetivo do projeto, que consiste em negar rapidamente imagens que certamente não possuem nenhuma face, e observando os resultados obtidos e análisados, pode-se concluir que tanto o algoritmo de Viola-Jones quanto o MTCNN poderiam ser utilizado de forma satisfatória para solucionar tal problema, mesmo utilizando um modelo já treinado, desde que parametrizados da forma correta, permitindo a obtenção lucros de alguns centavos por imagem, que em grande escala (milhares de imagens) pode ser relevante para uma empresa.

Durante o desenvolvimento do projeto foi possível perceber que o ajuste fino de um modelo tem grande impacto no seu resultado final, podendo ser considerada a parte mais importante para um projeto de implementação de um sistema de detecção facial.

Além disso, os testes apenas apresentaram resultados positivos após a seleção correta do conjunto de imagens, o que é um forte indicativo de que o modelo precisa ser preparado especificamente para as necessidades existentes.

\section{Trabalhos Futuros}

Para dar continuidade ao desenvolvimento desse projeto, podem ser estudadas outras metodologias de detecção facial ou evoluções da metodologia utilizada, adicionando técnicas como corte e rotação das imagens originais, que permitiriam ao classificador ser mais específico, por trabalhar com imagens melhor padronizadas.

Além disso, dada a necessidade apresentada, em que as imagens devem cumprir algumas características específicas além de apenas conter uma face, torna-se interessante a busca por um conjunto maior de imagens que permitiria o treinamento de um modelo específico para tal problema, que deve apresentar resultados ainda melhores.
