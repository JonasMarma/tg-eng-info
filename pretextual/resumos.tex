% ---
% RESUMOS
% ---

% RESUMO em português
\setlength{\absparsep}{18pt} % ajusta o espaçamento dos parágrafos do resumo
\begin{resumo}
  Devido aos grandes avanços na área de visão computacional, muitas ferramentas relacionadas ao assunto têm sido criadas, mas torna-se complexo escolher qual a ideal para a necessidade existente e como podem ser aplicadas. Para entender melhor as opções disponíveis e suas aplicações, este trabalho apresenta um estudo sobre o desempenho de dois diferentes algoritmos de detecção facial em imagens e a análise da viabilidade da aplicação dos mesmos em um projeto comercial. Foi estudado o método de Viola-Jones de análise de características e uma rede convolucional em cascata (\textit{MTCNN}) e o desempenho foi medido utilizando conjunto de imagens selecionado manualmente de acordo com uma lista de regras estabelecidas. A análise dos resultados foi feita em função da acurácia obtida e do lucro gerado em uma situação hipotética de uma empresa que necessita validar fotos cadastrais. Com o método de Viola-Jones foi demonstrado que é possível obter até \$0,136 de lucro por imagem analisada enquanto com o método \textit{MTCNN} é possível ter o resultado mais lucrativo com \$0,142 de lucro por imagem analisada. Após a análise, foi possível determinar que é comercialmente interessante a implementação de modelos de visão computacional mas é necessário dedicar esforços para adequação das ferramentas aos objetivos específicos de cada caso.

  \textbf{Palavras-chaves}: Aprendizado de Máquina. Detecção Facial. Visão Computacional.
\end{resumo}

% ABSTRACT in english
\begin{resumo}[Abstract]
  \begin{otherlanguage*}{english}
    Due to the great advances in the computer vision area, many tools related to this subject have been created, but it has become complex to choose which one is ideal for the existing needs and how they can be applied. In order to understand better the options available and their applications, this paper presents a study on the performance of two different face detection algorithms and the analysis of how feasible is to apply them in a commercial project. The Viola-Jones method of characteristic analysis and a cascade convolutional network (\textit{MTCNN}) were studied and their performances were measured using a set of manually selected images according to a list of established rules. The analysis of the results was made according to the accuracy obtained and the profit generated in a hypothetical situation of a company that needs to validate registration photos. It was demonstrated that Viola-Jones method can obtain up to \$0,136 profit per analyzed image while the \textit{MTCNN} method can obtain the most profitable result with \$0,142 profit per analyzed image. After the analysis, it was possible to determine that it is commercially interesting to implement computer vision models, but it is necessary to dedicate efforts to adapt the tools to the specific objectives of each case.

    \vspace{\onelineskip}

    \noindent
    \textbf{Keywords}: Machine Learning. Face Detection. Computer Vision.
  \end{otherlanguage*}
\end{resumo}