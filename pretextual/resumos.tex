% ---
% RESUMOS
% ---

% RESUMO em português
\setlength{\absparsep}{18pt} % ajusta o espaçamento dos parágrafos do resumo
\begin{resumo}
Previsões de demanda geralmente são feitas com base em dados históricos, sendo indicadores críticos para empresas que trabalham com logística porque influenciam em decisões e planejamentos futuros. Com as tecnologias atuais de informação, é possível ter acesso a uma quantidade enorme de dados, que podem ser processados de forma automática para gerar previsões mais confiáveis.

Este trabalho explorará diferentes algoritmos e combinações de dados, fazendo comparações da confiabilidade de previsões de demanda.

  \textbf{Palavras-chaves}: Séries temporais. Aprendizado de máquina. Previsão de Demanda.
\end{resumo}

\begin{comment}
% ABSTRACT in english
\begin{resumo}[Abstract]
  \begin{otherlanguage*}{english}
    Due to the great advances in the field of computer vision, many tools related to this subject have been created, but it has become complex to choose which one is ideal for the existing needs and how they can be applied. In order to understand better the options available and their applications, this paper presents a study on the performance of two different face detection algorithms and the analysis of how feasible is to apply them in a commercial project. The Viola-Jones method of characteristic analysis and a Multi-Task Cascaded Convolutional Neural Network (MTCNN) were studied and their performances were measured using a set of manually selected images according to a list of established rules. The analysis of the results was made according to the accuracy obtained and the profit generated in a hypothetical situation of a company that needs to validate registration photos. It was demonstrated that Viola-Jones method can obtain profitable result and with accuracy of 93,6\%, while the MTCNN method can obtain the most profitable result with accuracy of 94,2\%. After the analysis, it was possible to determine that it is commercially interesting to implement computer vision models, but it is necessary to dedicate efforts to adapt the tools to the specific objectives of each case.

    \vspace{\onelineskip}

    \noindent
    \textbf{Keywords}: Machine Learning. Face Detection. Computer Vision.
  \end{otherlanguage*}
\end{resumo}
\end{comment}